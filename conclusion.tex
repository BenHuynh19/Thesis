\section{Conclusions}

The paper has compared the performances of factor and sector diversification based upon the statistical test of Basak, Jagannathan, and Sun (2002). Successfully replicating factor exposures from individual stocks may be challenging (Ang et al. (2009)), and not all investors are willing to tediously undertake that work, this study, instead, leveraged the conveniences of accessible exchange traded funds (ETFs) as proxies for both factor and sector investing.\\
According to the mean-variance efficient test (BJS 2002), factor investing outperformed not only the S&P500 index, but also the sector diversification, both superior performances were proved with statistical significances for the full sample period. Since “performances of multifactor portfolios are more crisis-sensitive than those of passive portfolios” (Briere and Szafarz 2021), this research further investigated whether that observation was true during the highly volatile period: the pandemic 2020. The results were not statistically significant, however, the sign of the tests indicated that indeed factor investing was superior compared to sector investing. \\
Results from the study provided practical implementations for investors. Specifically, optimally combining different factors into a portfolio not only delivers the risk premiums promised by those strategies, but also potentially outperformed sector diversification and the S&P500 index during normal times and even market downturns, which were mostly considered as the times when those latter approaches were traditionally preferred.
