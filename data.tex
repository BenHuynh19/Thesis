\section{Data}

The primary objective of the paper is to provide practical comparison about investable portfolios, particularly a list of sector ETFs versus a set of factor ETFs. The result from this approach would be more convenient than Briere and Szafarz (2021) which constructed sector and factor indexes by all traded stocks in the US market. The latter case required investors to collect and adjust individual stocks manually to replicate the paper’s results while this time consuming, tedious work can be implemented directly through vastly available ETFs in the market.\\
All selected ETFs are designed by Blackrock – the world’s largest asset manager, with more than US 8 trillion assets under management (December 2020). The historical monthly returns are retrieved from Center for Research in Security Prices (CRSP)\footnote{I thank NEOMA Business School for granting me access to this database}  for the period from August 2013 to December 2020, including:
\begin{enumerate}[label=(\alph*)]
\item 12 sector ETFs:

\begin{itemize}
 \item iShares U.S. Utilities ETF (ticker IDU) with objective: “seeks to track the investment results of an index composed of U.S. equities in the utilities sector, including electricity, gas, and water”.
\item iShares U.S. Consumer Discretionary ETF (ticker IYC) with objective: “seeks to track the investment results of an index composed of U.S. equities in the consumer discretionary sector, including food, drugs, general retail items, and media”.
\item iShares U.S. Financials ETF (ticker IYF) with objective: “seeks to track the investment results of an index composed of U.S. equities in the financial sector, including banks, insurers, and credit card companies”.
\item iShares U.S. Financial Services ETF (ticker IYG) with objective: “seeks to track the investment results of an index composed of U.S. equities in the financial services sector, including investment banks, commercial banks, asset managers, credit card companies, and securities exchanges”.
\item iShares U.S. Healthcare ETF (ticker IYH) with objective: “seeks to track the investment results of an index composed of U.S. equities in the healthcare sector, including healthcare equipment and services, pharmaceuticals, and biotechnology companies”.
\item iShares U.S. Industrials ETF (ticker IYJ) with objective: “seeks to track the investment results of an index composed of U.S. equities in the industrials sector, including companies that produce goods used in construction and manufacturing”.
\item iShares U.S. Consumer Staples ETF (ticker IYK) with objective: “seeks to track the investment results of an index composed of U.S. equities in the consumer staples sector, including companies that produce a wide range consumer goods, automobiles, and household goods”.
\item iShares U.S. Basic Materials ETF (ticker IYM) with objective: “seeks to track the investment results of an index composed of U.S. equities in the basic materials sector, including companies involved with the production of raw materials, metals, chemicals and forestry products”.
\item iShares U.S. Real Estate ETF (ticker IYR) with objective: “seeks to track the investment results of an index composed of U.S. equities in the real estate sector, including real estate companies and REITs, which invest in real estate directly and trade like stocks”.
\item iShares U.S. Transportation ETF (ticker IYT) with objective: “seeks to track the investment results of an index composed of U.S. equities in the transportation sector, including airline, railroad, and trucking companies”.
\item iShares U.S. Technology ETF (ticker IYW) with objective: “seeks to track the investment results of an index composed of U.S. equities in the technology sector, including electronics, computer software and hardware, and informational technology companies”.
\item iShares U.S. Telecommunications ETF (ticker IYZ) with objective: “seeks to track the investment results of an index composed of U.S. equities in the telecommunications sector, including companies that provide telephone and internet products, services, and technologies”.
	\end{itemize}

\item 7 factor ETFs:
\begin{itemize}
\item iShares MSCI USA Size Factor ETF (ticker SIZE) with objective: “seeks to track the investment results of an index composed of U.S. large- and mid-capitalization stocks with relatively smaller average market capitalization”.
\item iShares MSCI USA Min Vol Factor ETF (ticker USMV) with objective: “seeks to track the investment results of an index composed of U.S. equities that, in the aggregate, have lower volatility characteristics relative to the broader U.S. equity market”.
\item iShares MSCI Emerging Markets Min Vol Factor ETF (ticker EEMV) with objective: “seeks to track the investment results of an index composed of emerging market equities that, in the aggregate, have lower volatility characteristics relative to the broader emerging equity markets
\item iShares MSCI EAFE Min Vol Factor ETF (ticker EFAV) with objective: “seeks to track the investment results of an index composed of developed market equities that, in the aggregate, have lower volatility characteristics relative to the broader developed equity markets, excluding the U.S. and Canada, including stocks in Europe, Australia, Asia and the Far East with potentially less risk”.
\item iShares MSCI USA Momentum Factor ETF (ticker MTUM) with objective: “seeks to track the investment results of an index composed of U.S. large- and mid-capitalization stocks exhibiting relatively higher price momentum”
\item iShares MSCI USA Quality Factor ETF (ticker QUAL) with objective: “seeks to track the investment results of an index composed of U.S. large- and mid-capitalization stocks with quality characteristics as identified through certain fundamental metrics, including high return on equity, stable year-over-year earnings growth and low financial leverage”.
\item iShares MSCI USA Value Factor ETF (ticker VLUE) with objective: “seeks to track the investment results of an index composed of U.S. large- and mid-capitalization stocks with value characteristics and relatively lower valuations based on fundamentals”.
	\end{itemize}

\item Treasury Bill (T-bill) with 30 days to maturity is treated as the risk free rate

\item And lastly, returns on the Standard & Poor's Composite Index (S&P500) is considered as the market portfolio

\end{enumerate}


