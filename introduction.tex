\section{Introduction}

In modern portfolio management, the act of grouping various assets into a portfolio generally serves for either diversification benefits or risk premiums. The former has been notably established in the field, especially through sector indices to reduce the overall volatility of the portfolio (De Moor & Sercu, 2011), the latter, however, has maintained one of the most dynamic research subject over the years, starting from the birth of CAPM to Fama-French 5 factor models (FF5) and the “factor zoo”\footnote{The term was introduced by Cochrane (2011) in describing a great number of available factors regarding asset pricing models}  as of nowadays.
Theoretically speaking, stock returns are primarily stemmed from risk premiums that they are exposed to. This logic was a bedrock for active investors whose jobs was to exploit any inefficiencies in the financial market. An active approach provides portfolio managers flexible maneuverability in selecting specific stocks in an effort to outperform a pre-defined benchmark. Adversely, a passive approach generally tracks the performance of an index by constructing a portfolio that has the same underlying stocks of the index. While the former tries to gain extra excess returns based upon asset miss-pricing, the latter believes wholly in the efficiency of the market performance in general. 
High fees are the mostly criticized characteristic of active investing, practically and theoretically, while this aspect is relatively low for passive approach. Specifically, in an effort to replicate the risk premiums from original factors, fund managers typically have to scan through the entire market, then classify certain stocks into proper portfolios based upon their performances. Not surprisingly, this whole process is time consuming and difficult, not only tediously requires labor works but also remarkably desires professional knowledge for sensible alterations. Though, the critical reason of the high fee is due to transaction costs, which arise when those managers have to frequently re-balance their portfolio weights to reflect updated fluctuations of all the traded stocks. These features mostly explain high fees charged by those active funds. The birth of ETF, however, has somewhat solved these issues since the prices of traded ETF already reflected the re-balancing costs of underlying baskets. Furthermore, investors can trade these ETFs simultaneously as normal stocks, hence the possibility of grouping them into a diversified portfolio may be promising, not only regarding to the potential of risk premiums, but also the diversification benefits that these ETFs may offer.
The natural question, therefore, is that can these multifactor portfolios, comprising of traded ETFs, beat those traditional sector diversification? The study leveraged statistical properties of Basak, Jagannathan, and Sun (2002), and followed testing procedures suggested by Ehling and Ramos (2006) to compare direct performances of factor and sector ETFs for the period from August 2013 to December 2020, including the global pandemic year to consider its impact. The obvious advantage of adopting ETFs, instead of replicating original factors based upon all traded stocks, is that results from the study will be easily applied by all kind of investors since grouping ETFs is as simple as grouping individual stocks. Furthermore, in adapting to this highly volatile and fluctuated market, an ease of re-balancing portfolio's weights would be another great asset over the traditional approach in constructing factors from scratch, not to mention the lower transaction costs and time consuming of the latter. 
